\documentclass[12pt,pdftex,a4paper]{article}
\usepackage[english]{babel}
\usepackage{enumitem} 
\usepackage{amsmath}
\usepackage{amssymb}
\usepackage{bbm}
\usepackage[utf8]{inputenc}
\usepackage{float}
\usepackage{cleveref}

\newcommand{\bbN}{\mathbbm{N}}
\newcommand{\bbR}{\mathbbm{R}}
\newcommand{\bbZ}{\mathbbm{Z}}
\newcommand{\bbI}{\mathbbm{I}}

\usepackage{amsthm}
\usepackage{amsfonts}
\usepackage{stmaryrd}
\usepackage{mathtools}
\usepackage{esvect} % Schöne Vektorpfeile mit \vv{\alpha}
\usepackage[usenames]{color}
\usepackage{polynom}
\usepackage{geometry}
\usepackage{tikz}
\usetikzlibrary{decorations.pathreplacing}
\geometry{verbose,a4paper,tmargin=25mm,bmargin=25mm,lmargin=15mm,rmargin=15mm}
\usepackage{graphicx}
\makeatletter
\def\ScaleIfNeeded{%
\ifdim\Gin@nat@width>\linewidth
\linewidth
\else
\Gin@nat@width
\fi
}
\makeatother

%\geometry{verbose,a4paper,tmargin=25mm,bmargin=25mm,lmargin=15mm,rmargin=20mm}
 
\title{Mathe}

\newcommand{\setN}[0]{\mathbb{N}}
\newcommand{\setF}[0]{\mathbb{F}}
\newcommand{\setR}[0]{\mathbb{R}}
\newcommand{\setZ}[0]{\mathbb{Z}}
\newcommand{\setC}[0]{\mathbb{C}}
\newcommand{\setP}[0]{\mathbb{P}}
\newcommand{\setQ}[0]{\mathbb{Q}}
\newcommand{\setK}[0]{\mathbb{K}}
\newcommand{\winkel}{<\hspace{-1.25ex})\hspace{2.25ex}}
\newcommand{\axi}[1]{{\label{#1}(#1)}}
\newtheorem{defi}{Definition}[section]
\newtheorem{satz}[defi]{Satz}
\newtheorem{prop}[defi]{Proposition}
\newtheorem{koro}[defi]{Korollar}
\newtheorem{lemma}[defi]{Lemma}
\newtheorem*{bsp}{Beispiel}
\newtheorem*{commen}{Bemerkung}
\newenvironment{alphb}{\begin{enumerate}
\def\theenumi{(\alph{enumi})}}{\end{enumerate}}
\newenvironment{arabb}{\begin{enumerate}
\def\theenumi{\arabic{enumi})}}{\end{enumerate}}
\renewcommand{\labelenumi}{\theenumi}
\renewcommand{\theenumi}{\arabic{enumi}.}
\newcommand{\litoinf}{\lim\limits_{n\to\infty}}
\newcommand{\sumtoinf}{\sum\limits_{n=0}^\infty}
\newcommand{\sumtok}{\sum\limits_{n=0}^k}
\newcommand{\sumin}{\sum\limits^\infty}
\newcommand{\fol}{_{n\in\setN}}
\newcommand{\dx}{\mathrm{d}x}
\newcommand{\dt}{\mathrm{d}t}
\usepackage{tabulary}
\usepackage{pdfpages}
\usepackage{todonotes}

\DeclareMathOperator{\Span}{Span}
\let\oldphi\phi
\renewcommand \phi \varphi
\newcommand \my \mu
\definecolor{dunkelgruen}{rgb}{0,0.4,0}


%\usepackage[pdftex]{graphicx}
\usepackage{listings}
\lstset{language=Python,basicstyle=\footnotesize}

\usepackage{newunicodechar}
\newunicodechar{°}{\deg} % \deg wird zum °-Zeichen für Winkel oder Temperaturen. kA, warum LaTeX das nicht direkt mag.
\begin{document}
\title{Bachelor-Forschungsprojekt Informatik:\\Relevante OSM-Tags vorschlagen}
\author{Marco Hildebrand, XXXX, stXXXX@stud.uni-stuttgart.de\\
		Lukas Baur, XXX, stXXX@stud.uni-stuttgart.de\\
		Felix Bühler, 2973410, st117123@stud.uni-stuttgart.de}
\maketitle
%%%%%%%%%%%%%%%%%%%%%%%%%%%%%%%%%%%%%%%%%%%%%%%%
%Damit unser armer Tutor nicht mehr unsere Handschrift entziffern muss, machen wir unsere Abgaben nun in \LaTeX -- ist das nicht nett von uns?
%%%%%%%%%%%%%%%%%%%%%%%%%%%%%%%%%%%%%%%%%%%%%%%%
\section{Einleitung}
\section{Vorgehensweise}
Anschauen von wiki xml dump

unbrauchbare daten, da viel untereinander verlinkt ist.

herunterladen der tags: https://taginfo.openstreetmap.org/

\section{Gettings started}
\subsection{languages}
einfach eine liste aller sprachen bekommen mithilfe \texttt{taginfo-wiki.db}.

Die kann man von \texttt{https://taginfo.openstreetmap.org/download} herunterladen.

\subsection{export-links}
herunterladen der osm-wiki sitemap
\texttt{https://wiki.openstreetmap.org/sitemap-index-wiki.xml}


davon interessiert uns nur \texttt{sitemap-wiki-NS\_0-0.xml} der rest enthält daten zu den nutzern, diskussionen und historie

\section{crawl}
alle gesammelten link in die \texttt{links.txt} legen

\texttt{scrapy crawl osmWiki -t json -o keys.json}

\subsection{pretty json}
\texttt{python -m json.tool keys.json > keys-pretty.json}


\end{document}
